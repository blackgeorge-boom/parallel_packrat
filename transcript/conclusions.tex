\chapter{ Συμπεράσματα }
\label{ch:conclusions}

Οι Parsing Expression Grammars αποτελούν διαισθητικά ένα κατάλληλο εργαλείο προσδιορισμού γραμματικών, ιδιαίτερα αν συγκριθούν με τις γραμματικές χωρίς συμφραζόμενα. 
Αρχικά, ο σχεδιαστής της γραμματικής είναι ευκολότερο να σκέφτεται πώς αναλύεται μία δοσμένη συμβολοσειρά στα συστατικά της, σε σχέση με το πώς θα γεννηθεί η συμβολοσειρά μέσα από τους κανόνες της γραμματικής (στο πνεύμα των Context Free γραμματικών).
Επιπλέον, ο ίδιος ο ορισμός των PEGs ορίζει απευθείας και τον αντίστοιχο συντακτικό αναλυτή της γραμματικής αυτής.

Ένας γεννήτορας συντακτικών αναλυτών είναι ένα βολικό εργαλείο για να φτιάχνουμε αυτόματα αναλυτές για μία γραμματική PEG που έχουμε ορίσει τυπικά.
Ο γεννήτορας παίρνει ως είσοδο μία τυπική περιγραφή της PEG και δίνει ως έξοδο ένα στιγμιότυπο της γραμματικής αυτής που μπορεί να αναλύσει ένας packrat parser.
Ο γεννήτορας πρακτικά δημιουργεί γραμματικές αλλά, όπως είπαμε, η κατασκευή μίας PEG συνεπάγεται και την κατασκευή του αντίστοιχου συντακτικού αναλυτή.
Επομένως, ο γεννήτορας κατασκευάζει συντακτικούς αναλυτές.

Ιδιαίτερα, λοιπόν, αν ο προκύπτων συντακτικός αναλυτής είναι και αποδοτικός στο χρόνο και στο χώρο, ένα τέτοιο εργαλείο θα ήταν βολικό για έναν προγραμματισή ο οποίος πειραματίζεται με μία δική του γραμματική για κάποιον ειδικό σκοπό.
Διότι, έχει τη δυνατότητα να ορίσει και να αλλάζει εύκολα τον ορισμό της γραμματικής του, αλλά και να παίρνει έναν γρήγορο συντακτικό αναλυτή για τις εφαρμογές του.

Από το προηγούμενο κεφάλαιο φαίνεται μάλλον ότι το packrat parsing με ελαστικό κυλιόμενο παράθυρο αποτελεί την καλύτερη βελτίωση του packrat, καθώς χρειάζεται αισθητά λιγότερο χρόνο εκτέλεσης, αλλά και σταθερή μνήμη, εξοβελίζοντας ένα κυρίαρχο μειονέκτημα του κλασικού αλγορίμου.


