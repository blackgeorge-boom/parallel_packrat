\chapter{ Συμπεράσματα }
\label{ch:conclusions}

Οι Parsing Expression Grammars αποτελούν διαισθητικά ένα κατάλληλο εργαλείο προσδιορισμού γραμματικών, ιδιαίτερα αν συγκριθούν με τις γραμματικές χωρίς συμφραζόμενα. 
Αρχικά, ο σχεδιαστής της γραμματικής είναι ευκολότερο να σκέφτεται πώς αναλύεται μία δοσμένη συμβολοσειρά στα συστατικά της, σε σχέση με το πώς θα γεννηθεί η συμβολοσειρά μέσα από τους κανόνες της γραμματικής (στο πνεύμα των Context Free γραμματικών).
Επιπλέον, ο ίδιος ο ορισμός των PEGs ορίζει απευθείας και τον αντίστοιχο συντακτικό αναλυτή της γραμματικής αυτής.

Ένας γεννήτορας συντακτικών αναλυτών είναι ένα βολικό εργαλείο για να φτιάχνουμε αυτόματα αναλυτές για μία γραμματική PEG που έχουμε ορίσει τυπικά.
Ο γεννήτορας παίρνει ως είσοδο μία τυπική περιγραφή της PEG και δίνει ως έξοδο ένα στιγμιότυπο της γραμματικής αυτής που μπορεί να αναλύσει ένας packrat parser.
Ο γεννήτορας πρακτικά δημιουργεί γραμματικές αλλά, όπως είπαμε, η κατασκευή μίας PEG συνεπάγεται και την κατασκευή του αντίστοιχου συντακτικού αναλυτή.
Επομένως, ο γεννήτορας κατασκευάζει συντακτικούς αναλυτές.

Ιδιαίτερα, λοιπόν, αν ο προκύπτων συντακτικός αναλυτής είναι και αποδοτικός στο χρόνο και στο χώρο, ένα τέτοιο εργαλείο θα ήταν βολικό για έναν προγραμματιστή ο οποίος πειραματίζεται με μία δική του γραμματική για κάποιον ειδικό σκοπό.
Διότι, έχει τη δυνατότητα να ορίσει και να αλλάζει εύκολα τον ορισμό της γραμματικής του, αλλά και να παίρνει έναν γρήγορο συντακτικό αναλυτή για τις εφαρμογές του.

Από το προηγούμενο κεφάλαιο φαίνεται μάλλον ότι το packrat parsing με ελαστικό κυλιόμενο παράθυρο αποτελεί την καλύτερη βελτίωση του packrat, καθώς χρειάζεται αισθητά λιγότερο χρόνο εκτέλεσης, αλλά και σταθερή μνήμη, εξοβελίζοντας ένα κυρίαρχο μειονέκτημα του κλασικού αλγορίμου.

Σχετικά με τον παράλληλο αλγόριθμο, δεν παρατηρείται επιτάχυνση (speedup) σε σχέση με τη σειριακή περίπτωση.
Ενώ υπάρχει βελτίωση για expr\_limit από 2 σε 4 και, ίσως, σε 6 (που ισοδυναμεί εν δυνάμει με αντιστοίχως παραπάνω νήματα), δεν υπάρχει βελτίωση σε σχέση με τη σειριακή περίπτωση.
Από την άλλη, μέγιστο βάθος του δέντρου των κλήσεων μεγαλύτερο του 1 μάλλον δυσχεραίνει την κατάσταση, καθώς αφήνει χώρο για παραπάνω νήματα από όσα μπορεί να διαθέσει ταυτόχρονα το μηχάνημα, οπότε αρκετά νήματα μένουν ανενεργά.

Στη ρίζα του, το πρόβλημα φαίνεται να είναι πως η πράξη της διατεταγμένης επιλογής, παρόλο που θεωρητικά αφήνει χώρο για παράλληλη εκτέλεση, δεν συμπεριφέρεται τόσο καλά στην πράξη.
Συγκεκριμένα, φαίνεται ότι το κόστος της δημιουργίας και της αναμονής των νημάτων είναι μεγαλύτερο από το κέρδος του διαμοιρασμού του φόρτου εργασίας.
Ειδικά στην αναμονή, όταν τα εναπομείναντα νήματα πρέπει να τερματιστούν, αυτό δεν μπορεί να γίνει άμεσα, αλλά πρέπει να τους σταλεί σήμα και να το λάβουν σε κάποιο σημείο του κώδικα.
Αυτό συνεπάγεται ότι πιθανώς εκτελούνται και υπολογισμοί ενδιάμεσων αποτελεσμάτων από νήματα που δεν έχουν προλάβει να δουν το σήμα τερματισμού, οι οποίοι ούτε ωφελούν, αλλά αντίθετα καθυστερούν το αρχικό νήμα-γονέα.

Ακόμη, ενδέχεται στη γραμματική που εξετάσαμε στα πειράματα να είναι συχνό φαινόμενο στη διατεταγμένη επιλογή, να πετυχαίνει κάποια από τις πρώτες επιλογές (αν όχι η πρώτη).
Σε αυτήν την περίπτωση, η δημιουργία νημάτων και για τις επόμενες υποεκφράσεις είναι μάταια, ενώ προσθέτει και καθυστέρηση (overhead) στον συνολικό αλγόριθμο, οπότε η εκτέλεση γίνεται χειρότερη και από τη σειριακή.

Στο \cite{Fowler2009} περιγράφεται μία απόπειρα για παραλληλοποίηση του αλγορίθμου στη διάσταση της εισόδου.
Δηλαδή, χωρίζεται η είσοδος σε μπλοκς τα οποία μοιράζονται μεταξύ νημάτων-εργατών, τα οποία υπολογίζουν με κάποια ευριστική μέθοδο διάφορα κελιά μέσα σε αυτά τα μπλοκς.
Στο πρώτο μπλοκ, όμως, υπάρχει ένα κύριο νήμα που ξεκινάει από την αρχή να υπολογίζει κελιά με τον σειριακό αλγόριθμο.
Ωστόσο, μόλις φτάσει στο σύνορο του μπλοκ με το γειτονικό νήμα-εργάτη, του "παραδίδει" μέχρι εκείνο το σημείο την υποέκφραση την οποία έχει καταφέρει να αναλύσει.
Οπότε, το επόμενο νήμα, έχοντας υπολογίσει κάποια αποτελέσματα ήδη, ξεκινάει από την υποέκφραση και αρχίζει να την αναλύει με το σειριακό αλγόριθμο, ελπίζοντας ότι κάποια από τα νήματα που υπολόγισε ως τότε θα χρησιμεύσουν.
Στη συνέχεια παραδίδει τη δική του υποέκφραση (όσο την έχει προχωρήσει) στο νήμα-εργάτη του επόμενου μπλοκ και ούτω καθεξής. Για μία απλή γραμματική (τη γραμματική που περιγράφει τις PEGs) ο ισχυρισμός είναι ότι επετεύχθη επιτάχυνση περίπου κατά 2.5, αν και η υλοποίηση ήταν μακράν πιο πολύπλοκη από τη δική μας ιδέα.

Πάντως, για τη βελτίωση με τη βοήθεια πολλών πυρήνων, στην περίπτωση που πρέπει να αναλυθεί συντακτικά ένα μεγάλο σύνολο αρχείων (π.χ. ενός μεγάλου project), θα μπορούσε να χρησιμοποιηθεί και ένας πυρήνας ανά αρχείο για συντακτική ανάλυση, ώστε να βελτιωθεί ο χρόνος εκτέλεσης.
Όπως και να 'χει, συνολικά, για τη βελτίωση της επίδοσης του packrat μάλλον το ελαστικό κυλιόμενο παράθυρο αποτελεί την καλύτερη (και απλούστερη) επιλογή.

