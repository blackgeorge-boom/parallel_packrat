\documentclass[diploma]{softlab-thesis}

%%%
%%%  Add and configure the packages that you need for your thesis
%%%

\usepackage{minted}


%%%
%%%  The document
%%%

\begin{document}

%%%  Title page

\frontmatter

\title{Σχεδίαση και Υλοποίηση μιας Καταπληκτικής Γλώσσας Προγραμματισμού}
\author{Γεράσιμος Τ. Ιωάννου}
\authoren{Gerasimos T. Ioannou}
\date{Αύγουστος 2019}
\datedefense{26}{8}{2019}

\supervisor{Νικόλαος Σ. Παπασπύρου}
\supervisorpos{Καθηγητής Ε.Μ.Π.}

\committeeone{Νικόλαος Σ. Παπασπύρου}
\committeeonepos{Καθηγητής Ε.Μ.Π.}
\committeetwo{Πέτρος Παπαδόπουλος}
\committeetwopos{Επίκ. Καθηγητής Ε.Μ.Π.}
\committeethree{Γεώργιος Νικολάου}
\committeethreepos{Αν. Καθηγητής Ε.Κ.Π.Α.}

\TRnumber{CSD-SW-TR-42-17}  % number-year, ask nickie for the number
\department{Τομέας Τεχνολογίας Πληροφορικής και Υπολογιστών}

\maketitle


%%%  Abstract, in Greek

\begin{abstractgr}%
  Σκοπός της παρούσας εργασίας είναι αφενός η σχεδίαση μίας απλής
  γλώσσας υψηλού επιπέδου με υποστήριξη για προγραμματισμό με
  αποδείξεις, αφετέρου η υλοποίηση ενός μεταγλωττιστή για τη γλώσσα
  αυτή που θα παράγει κώδικα για μία γλώσσα ενδιάμεσου επιπέδου
  κατάλληλη για δημιουργία πιστοποιημένων εκτελέσιμων.

  Στη σημερινή εποχή, η ανάγκη για αξιόπιστο και πιστοποιημένα ασφαλή
  κώδικα γίνεται διαρκώς ευρύτερα αντιληπτή. Τόσο κατά το παρελθόν όσο
  και πρόσφατα έχουν γίνει γνωστά προβλήματα ασφάλειας και
  συμβατότητας προγραμμάτων που είχαν ως αποτέλεσμα προβλήματα στην
  λειτουργία μεγάλων συστημάτων και συνεπώς οικονομικές επιπτώσεις
  στους οργανισμούς που τα χρησιμοποιούσαν. Τα προβλήματα αυτά
  οφείλονται σε μεγάλο βαθμό στην έλλειψη δυνατότητας προδιαγραφής και
  απόδειξης της ορθότητας των προγραμμάτων που χαρακτηρίζει τις
  σύγχρονες γλώσσες προγραμματισμού. Για το σκοπό αυτό, έχουν προταθεί
  συστήματα πιστοποιημένων εκτελέσιμων, στα οποία έχουμε τη δυνατότητα
  να προδιαγράφουμε την ορθότητα των προγραμμάτων, και να παρέχουμε
  μία τυπική απόδειξη αυτής, η οποία μπορεί να ελεγχθεί μηχανιστικά
  πριν το χρόνο εκτέλεσης.

  Τα συστήματα που έχουν προταθεί είναι ενδιάμεσου επιπέδου οπότε η
  διαδικασία προγραμματισμού σε αυτά είναι ιδιαίτερα πολύπλοκη. Οι
  γλώσσες υψηλού επιπέδου που συνοδεύουν αυτά τα συστήματα, ενώ είναι
  ιδιαίτερα εκφραστικές, παραμένουν δύσκολες στον προγραμματισμό.  Μία
  απλούστερη γλώσσα υψηλού επιπέδου, όπως αυτή που προτείνουμε σε αυτή
  την εργασία, θα επέτρεπε ευρύτερη εξάπλωση του συγκεκριμένου
  ιδιώματος προγραμματισμού.

  Στη γλώσσα που προτείνουμε, ο προγραμματιστής προδιαγράφει τη μερική
  ορθότητα του προγράμματος, δίνοντας προσυνθήκες και μετασυνθήκες για
  τις παραμέτρους και τα αποτελέσματα των συναρτήσεων που ορίζει.
  Επίσης δίνει ένα σύνολο θεωρημάτων βάσει του οποίου κατασκευάζονται
  αποδείξεις της ορθής υλοποίησης και χρήσης των συναρτήσεων αυτών. Ως
  μέρος της εργασίας, έχουμε υλοποιήσει σε γλώσσα OCaml ένα
  μεταφραστή αυτής της γλώσσας στο σύστημα πιστοποιημένων
  εκτελέσιμων NFLINT.

  Επιτύχαμε να διατηρήσουμε τη γλώσσα κοντά στο ύφος των ευρέως
  διαδεδομένων συναρτησιακών γλωσσών, καθώς και να διαχωρίσουμε τη
  φάση προγραμματισμού από τη φάση απόδειξης της ορθότητας των
  προγραμμάτων. Έτσι ένας μέσος προγραμματιστής μπορεί εύκολα να
  προγραμματίζει στη γλώσσα που προτείνουμε με τον τρόπο που ήδη
  γνωρίζει, και ένας γνώστης μαθηματικής λογικής να αποδεικνύει σε
  επόμενη φάση την μερική ορθότητα των προγραμμάτων. Ως απόδειξη της
  πρακτικότητας της προσέγγισης αυτής, παραθέτουμε ένα σύνολο
  παραδειγμάτων στη γλώσσα με απόδειξη μερικής ορθότητας.
\begin{keywordsgr}
  Γλώσσες προγραμματισμού,
  Προγραμματισμός με αποδείξεις,
  Ασφαλείς γλώσσες προγραμματισμού,
  Πιστοποιημένος κώδικας.
\end{keywordsgr}
\end{abstractgr}


%%%  Abstract, in English

\begin{abstracten}%
  The purpose of this diploma dissertation is on one hand the design
  of a simple high-level language that supports programming with
  proofs, and on the other hand the implementation of a compiler for
  this language. This compiler will produce code for an
  intermediate-level language suitable for creating certified
  binaries.

  The need for reliable and certifiably secure code is even more
  pressing today than it was in the past. In many cases, security and
  software compatibility issues put in danger the operation of large
  systems, with substantial financial consequences. The lack of a
  formal way of specifying and proving the correctness of programs that
  characterizes current programming languages is one of the main reasons
  why these issues exist. In order to address this problem, a number of
  frameworks with support for certified binaries have recently been
  proposed. These frameworks offer the possibility of specifying and
  providing a formal proof of the correctness of programs. Such a proof
  can easily be checked for validity before running the program.

  The frameworks that have been proposed are intermediate-level in
  nature, thus the process of programming in these is rather cumbersome.
  The high-level languages that accompany some of these frameworks,
  while very expressive, are hard to use. A simpler high-level language,
  like the one proposed in this dissertation, would enable further use
  of this programming idiom.

  In the language we propose, the programmer specifies the partial
  correctness of a program by annotating function definitions with pre-
  and post-conditions that must hold for their parameters and results.
  The programmer also provides a set of theorems, based on which proofs
  of the proper implementation and use of the functions are constructed.
  An implementation in OCaml of a compiler from this language to the
  NFLINT certified binaries framework was also completed as part of this
  dissertation.

  We managed to keep the language close to the feel of the current
  widespread functional languages, and also to fully separate the
  programming stage from the correctness-proving stage. Thus an average
  programmer can program in a familiar way in our language, and later an
  expert on formal logic can prove the semi-correctness of a program.
  As evidence of the practicality of our design, we provide a number of
  examples in our language with full semi-correctness proofs.
\begin{keywordsen}
  Programming languages,
  Programming with proofs,
  Secure programming languages,
  Certified code.
\end{keywordsen}
\end{abstracten}


%%%  Acknowledgements

\begin{acknowledgementsgr}
  Ευχαριστώ θερμά τον επιβλέποντα καθηγητή αυτής της διατριβής,
  κ.~Νίκο Παπασπύρου, για τη συνεχή καθοδήγηση και εμπιστοσύνη
  του. Ευχαριστώ επίσης τα μέλη της συμβουλευτικής επιτροπής,
  κ.κ.~Νίκο Παπαδόπουλο και Γιώργο Νικολάου για την πρόθυμη και
  πάντα αποτελεσματική βοήθειά τους, τις πολύτιμες συμβουλές και
  τις χρήσιμες συζητήσεις που είχαμε.  Θέλω να ευχαριστήσω ακόμα
  τον συμφοιτητή και φίλο Πέτρο Πετρόπουλο, ο οποίος με βοήθησε σε
  διάφορα στάδια αυτής της εργασίας.  Θα ήθελα τέλος να ευχαριστήσω
  την οικογένειά μου και κυρίως τους γονείς μου, οι οποίοι με
  υποστήριξαν και έκαναν δυνατή την απερίσπαστη ενασχόλησή μου τόσο
  με την εκπόνηση της διπλωματικής μου, όσο και συνολικά με τις
  σπουδές μου, καθώς και το Νίκο Καζαντζάκη που αποτέλεσε για μένα
  μοναδική πηγή έμπνευσης, όταν δεν ήξερα τι να γράψω στην εισαγωγή
  της εργασίας μου.
\end{acknowledgementsgr}

\begin{acknowledgementsen}
  I would like to thank all the people who supported my work and helped me get
  results of better quality.  I am also grateful to the supervisor of my thesis
  and the members of my committee for their patience and support in overcoming
  numerous obstacles I have been facing through my research.  I would also like
  to thank my fellow students for their feedback, cooperation and of course
  friendship.

  Last but not least, I would like to thank my family, my parents and my
  sister, for supporting me spiritually throughout writing this thesis and
  my life in general, and Nikos Kazantzakis for providing the inspiration
  that I needed when I didn't know what to write in my introduction.
\end{acknowledgementsen}


%%%  Various tables

\tableofcontents
%\listoftables
\listoffigures
%\listofalgorithms


%%%  Main part of the book

\mainmatter

\chapter{Εισαγωγή}

Κάθεται ο Αγάς της Λυκόβρυσης στο μπαλκόνι του απάνω από την πλατεία του
χωριού, καπνίζει το τσιμπούκι του και πίνει ρακή. Ψιχαλίζει ήσυχα, τρυφερά,
και στα γυρτά χοντρά μουστάκια του, τα φρεσκοβαμμένα με καραμπογιά,
κρέμουνται λαμπυρίζοντας μερικές ψιχάλες· κι o Αγάς, ξαναμμένος από τη ρακή,
τις αναγλείφει να δροσερέψει. Δεξά του στέκεται όρθιος ο σεΐζης, ένας
θεόρατος άγριος ανατολίτης, αλλήθωρος και κακομούτσουνος, με την τρουμπέτα
του· ζερβά του κάθεται διπλοπόδι, απάνω σε βελουδένιο μαξιλάρι, ένα όμορφο
στρουμπουλό τουρκόπουλο, που του ανάβει κάθε τόσο το τσιμπούκι και του
γεμίζει ακατάπαυτα την κούπα του ρακή.

Μισοσφαλνάει τα μαχμουρλίδικα μάτια και χαίρεται o Αγάς τον απάνω κόσμο· όλα
τα~’καμε καλά ο Θεός, συλλογιέται, πετυχεμένο πράμα είναι ο κόσμος τούτος,
τίποτα δεν του λείπει: αν πεινάσεις, έχει ψωμί και κρέας κοκκινιστό και
πιλάφι με κανέλα· αν διψάσεις, έχει το αθάνατο νερό, τη ρακή· αν νυστάξεις,
έχει κάμει ο Θεός τον ύπνο, ένα κι ένα για τη νύστα· αν θυμώσεις, έχει κάμει
το βούρδουλα και τα πισινά του ραγιά· αν σε πιάσουν τα μεράκια σου, έχει
κάμει τον αμανέ. Κι αν θες να ξεχάσεις τα ντέρτια και τα βάσανα του κόσμου,
έχει κάμει το Γιουσουφάκι.

--- Μεγάλος μάστορας είναι ο Αλλάχ, μουρμούρισε συγκινημένος, μεγάλος
μάστορας, μερακλής· κόβει το μυαλό του· πως του ήρθε τώρα κι έκαμε τη ρακή
και το Γιουσουφάκι!

Βουρκώνουν τα μάτια του Αγά από τη θρησκευτική κατάνυξη κι από το πολύ
ρακοπότι. Σκύβει από το μπαλκόνι και καμαρώνει τους ραγιάδες του να
σουλατσέρνουν στην πλατεία φρεσκοξουρισμένοι, γιορτοντυμένοι, με τα κόκκινα
φαρδιά ζωνάρια, με τα νιοπλυμένα κοντοβράκια, με τα γαλάζια
τουζλούκια. Άλλοι φορούν φέσι, άλλοι σαρίκι, άλλοι σκούφο από αρνίσια
προβιά. Οι πιο ασίκηδες έχουν και στο αυτί τους ένα κλωνί βασιλικό ή ένα
τσιγάρο.


\section{Σκοπός της εργασίας}

Τρίτη της Λαμπρής, τώρα τέλεψε η λειτουργιά. Γλυκιά, τρυφερή μέρα, ήλιος,
ψιχαλίζει, μύρισαν οι λεμονανθοί, μπουμπουκιάζουν τα δέντρα, ανασταίνουνται
τα χόρτα, ανεβαίνει ο Χριστός από κάθε σβωλαράκι χώμα. Σουλατσάρουν οι
χριστιανοί στην πλατεία, σμίγουν οι φίλοι, ασπάζεται ο ένας τον άλλο, λεν
«Χριστός ανέστη» κι ύστερα καθίζουν στον καφενέ του Κωσταντή ή κάτω από το
μεγάλο πλάτανο στη μέση της πλατείας, παραγγέλνουν ναργιλέδες και καφέδες κι
αρχίζει, σαν την ψιλή βροχή, το γλυκό κουβεντολόι.

--- Τέτοια θα~’ναι κι η Παράδεισο, λέει ο Χαράλαμπος ο καντηλανάφτης· ήλιος
απαλός, σιγανή βροχούλα, λεμονιές ανθισμένες, ναργιλέδες και ψιλή κουβέντα
στους αιώνες των αιώνων.

Στην άλλην άκρα της πολιτείας, πίσω από τον πλάτανο, υψώνεται
φρεσκοασβεστωμένη, με το χαριτωμένο καμπαναριό της, η εκκλησία του χωριού, η
Σταύρωση του Χριστού. Η πόρτα της είναι σήμερα στολισμένη με βάγια και
δάφνες. Γύρα τρογύρα, τα μαγαζάκια και τα εργαστήρια του χωριού: Το
σαμαράδικο του αγριάνθρωπου Παναγιώταρου, που τον λεν και Γυψοφά· γιατί μια
φορά έφεραν στο χωριό το γύψινο αγαλματάκι του Μ.~Ναπολέοντα, και το~’φαε·
κι υστέρα έφεραν ένα άλλο, του Κεμάλ-πασά, και το~’φαε κι αυτό· κι ύστερα
έφεραν του Βενιζέλου, και το~’φαε κι αυτό. Δίπλα, το μπαρμπέρικο του Αντώνη
«Ο Ερωτόκριτος», κι απάνω στην πόρτα κρεμασμένη μια ταμπέλα, με μεγάλα
κόκκινα γράμματα του αιμάτου: «Εξέρχονται και οδόντες»! Πιο πέρα το χασάπικο
του κυρ Δημητρού του κουτσού: «Κεφαλάκια φρέσκα, η Ηρωδιάς»! Κάθε Σάββατο
σφάζει ένα μουσκάρι και, πριν το σφάξει, του χρυσώνει τα κέρατα, του
μπογιατίζει το κούτελο, του περνάει κόκκινες κορδέλες στο λαιμό και το
γυρίζει στο χωριό κουτσαίνοντας και τελαλίζει τις χάρες του. Και τέλος ο
περίφημος καφενές του Κωσταντή, μακρόστενος, δροσερός, που μοσχομυρίζει καφέ
και τουμπεκί και φασκόμηλο το χειμώνα. Και στον τοίχο του δεξά κρέμουνται,
καμάρι του χωριού, τρεις μεγάλες γυαλιστερές λιθογραφίες: η Γενοβέφα από τη
μια μεριά, μισόγυμνη μέσα σ’~ένα δάσο τροπικό· από την άλλη μεριά, η
βασίλισσα Βικτώρια, παχιά, γαλανομάτα, με τεράστιο στήθος παραμάνας· και στη
μέση, άγριος, με γκρίζα θυμωμένα μάτια, μ’~έναν αψηλό σκούφο αστρακάν, ο
Κεμάλ-πασάς.

Όλοι αγαθοί ανθρώποι, δουλευταράδες, καλοί νοικοκυραίοι, πλούσιο το χωριό,
κι ο Αγάς του καλός άνθρωπος κι αυτός, μερακλής, που πολύ αγαπάει τη ρακή,
τις βαριές μυρωδιές, μόσκο καί πατσουλί, και το αφράτο τουρκόπουλο, που
κάθεται ζερβά του, στο βελουδένιο μαξιλάρι. Κοιτάζει τώρα τους χριστιανούς,
όπως κοιτάζει ο βοσκός τα καλοθρεμμένα αρνοπρόβατά του και χαίρεται.

«Καλοί άνθρωποι, συλλογιέται, γέμισαν κι εφέτο το κελάρι μου πεσκέσια της
Λαμπρής --- τυριά, κουλούρες σουσαμωτές, τσουρέκια, κόκκινα αυγά... Ένας, ας
είναι καλά, μου~’φερε κι ένα λαγηνάκι χιώτικη μαστίχα για το Γιουσουφάκι
μου, να μασάει και να μυρίζει το στοματάκι του...»

Είπε, κι έριξε μια τρυφερή ματιά στο αγαδόπουλο που μασούσε μαστίχα, παχουλό
κι αποχαυνωμένο.


\section{Η γλώσσα προγραμματισμού C++}

Κι έτσι που συλλογιόταν το κελάρι του γεμάτο αγαθά, και σιγοψιχάλιζε, και
γυάλιζαν οι πέτρες, και τα κοκόρια άρχισαν να λαλούν, και δίπλα του,
κουλουριαστό στα πόδια του, το Γιουσουφάκι μασούσε μαστίχα και χτυπούσε
ευτυχισμένο τη γλώσσα του, ο Αγάς ένιωσε ξαφνικά την καρδιά του να
ξεχειλίζει, σήκωσε το λαιμό, έκαμε ν’~αρχίσει τον αμανέ, μα
βαρέθηκε. Γυρνάει στο σεΐζη του και του γνέφει να βαρέσει την τρουμπέτα, να
σωπάσει ο λαός· κι ύστερα γυρνάει ζερβά του:

--- Τραγούδησέ μου, Γιουσουφάκι, να~’χεις την ευκή μου, τραγούδησε μου το
«Ντουνιά ταμπίρ, ρουγιά ταμπίρ, αμάν, αμάν!», τραγούδησέ μου το, γιατί θα
πλαντάξω!

Το παχουλό παιδόπουλο αποτραβάει χωρίς βιάση τη μαστίχα από το στόμα του,
την κολνάει στο γυμνό γόνατό του, ακουμπάει τη δεξά του παλάμη στο μάγουλο
κι αρχίζει τον αγαπημένο αμανέ του Αγά του: «Κόσμος κι όνειρο είναι ένα,
αμάν, αμάν!»

Παθητικιά, ναζλίδικη, η φωνή ανέβαινε, κατέβαινε, γουργούριζε σαν
περιστέρα. Κι ο Αγάς έκλεισε τα μάτια, κι όλη την ώρα που βαστούσε ο αμανές
τόσο ήταν βαλαντωμένος που~’χε ξεχάσει να πιει.

--- Είναι στα κέφια του ο Αγάς, μουρμούρισε ο Κωσταντής σερβίροντας τους
καφέδες· ας είναι καλά η ρακή.

--- Ας είναι καλά το Γιουσουφάκι, είπε χαμογελώντας πικρά ο Γιαννακός, ο
πραματευτής και γραμματοφόρος του χωριού, με τα κάτασπρα κοντοστρόγγυλα
γένια και τα πουλίσια, αρπαχτικά μάτια.

--- Ας είναι καλά η μοίρα η στραβή που τον έκαμε αυτόν Αγά κι εμάς ραγιάδες,
μουρμούρισε ο αδερφός του παπά, ο Χατζη-Νικολής, που έκανε το δάσκαλο του
χωριού, ξερακιανός, με γυαλάκια και μ’~ένα χοντρό μυτερό καρύδι του λαιμού,
που ανεβοκατέβαινε, όταν μιλούσε.

\begin{figure}[t]
\setlength\partopsep{-\topsep}% adjusts vertical space after the listing
\begin{minted}[frame=lines,linenos,numbersep=5pt]{c++}
#include <iostream>
int main() {
  std::cout << "Γεια σου κόσμε!" << std::endl;
}
\end{minted}
\caption{Στο τσακίρ κέφι, ο Αγάς γράφει και λίγο κώδικα.%
  \label{fig:hello-greek}}
\end{figure}


\section{Δομή της εργασίας}

Πήρε φωτιά, θυμήθηκε τους προγόνους, αναστέναξε:

--- Μια φορά, εξακολούθησε, κρατούσαν τα χώματα τούτα οι δικοί μας, οι
Έλληνες, γύρισε ο τροχός κι ήρθαν οι Βυζαντινοί, Έλληνες κι αυτοί και
χριστιανοί, ξαναγύρισε ο τροχός κι ήρθαν οι Αγαρηνοί... Μα αναστήθηκε ο
Χριστός, παιδιά, θ’~αναστηθεί κι η πατρίδα! Κωσταντή, έλα κέρασε τα
παλικάρια!

Ωστόσο ο αμανές τέλεψε, το τουρκόπουλο ξανάβαλε στο στόμα του τη μαστίχα κι
άρχισε πάλι ν’~αναχαράζει, αποχαυνωμένο. Βάρεσε ξανά η τρουμπέτα, μπορούσαν
τώρα οι ραγιάδες να γελούν και να φωνάζουν λεύτερα.


\chapter{Θεωρητικό υπόβαθρο}

Ο καπετάν Φουρτούνας, ένας από τους πέντε δημογέροντες του χωριού, πρόβαλε
στην πόρτα του καφενέ. Αψηλός, κορμάτος, παλιός καραβοκύρης, που χρόνια
αλώνιζε τη Μαύρη Θάλασσα κουβαλώντας ρούσικο σιτάρι και κάνοντας
κοντραμπάντο. Τρίχα δεν είχε το πρόσωπο του· σπανός και μαυροκίτρινος,
ταγαριασμένος, με βαθιές ζαρωματιές, και τα μάτια του τα μικρά, τα
κατάμαυρα, σπίθιζαν. Γέρασε, γέρασε και το καράβι του, τσακίστηκε μια νύχτα
απόξω από την Τραπεζούντα, κι ο καπετάν Φουρτούνας, καραβοτσακισμένος,
μπουχτισμένος, γύρισε στο χωριό του να σουρώσει όσο μπορεί περισσότερη ρακή
και, σαν έρθει η ώρα η καλή, να γυρίσει το πρόσωπό του κατά τον τοίχο και να
πεθάνει. Πολλά είχαν δει τα μάτια του, βαρέθηκε· δε βαρέθηκε, κουράστηκε, μα
ντρέπουνταν να το μολοήσει.


\section{Σημασιολογία γλωσσών προγραμματισμού}

Φορούσε σήμερα τις αψηλές καπετανίστικες μπότες του, τον κίτρινό του μουσαμά
και το αρχοντικό καλπάκι από αληθινό αστρακάν. Κρατούσε και το αψηλό ραβδί
του του δημογέροντα. Δυο τρεις χωριανοί προσηκώθηκαν να τον καλέσουν να
πάρει μια ρακή.

--- Δεν έχω καιρό, παιδιά, μήτε για ρακή, είπε· Χριστός ανέστη! Πάω στο
αρχοντικό του παπά, όπου έχουμε σύναξη οι προεστοί. Σε μια ώρα να κοπιάσετε
κι όσοι από σας είστε καλεσμένοι· κάντε το σταυρό σας κι ελατέ, καλά το
κατέχετε, έχουμε σήμερα δουλειά. Κι ένας να πάει να φωνάξει τον Παναγιώταρο
το σαμαρά, με του διαόλου τα γένια· αυτόν τον έχουμε μεγάλη ανάγκη.

Σώπασε μια στιγμή, τα μάτια του έπαιξαν παμπόνηρα:

--- Αν δεν είναι σπίτι του, θα~’ναι στης χήρας, είπε κι όλοι ξέσπασαν στα
γέλια.

Μα ο γερο-Χριστοφής ο αγωγιάτης, που~’χε μάθει στα νιάτα του, κι ας το
πλέρωσε ακριβά, τι θα πει σεβντάς, πετάχτηκε απάνω:

--- Τι γελάτε, μωρέ σερσέμηδες; φώναξε. Καλά κάνει· φωτιά στα τόπια, μωρέ
Παναγιώταρε, και μην τους ακούς! Λίγη είναι η ζωή, πολύς ο θάνατος, βίρα!

Ο χοντρο-Δημητρός ο χασάπης κούνησε το φρεσκοξουρισμένο κεφάλι:

--- Ο Θεός να~’χει καλά τη χήρα την Κατερίνα μας, είπε. Ο διάολος ξέρει από
τι κέρατα μας γλιτώνει!

Ο καπετάν Φουρτούνας γέλασε.

--- Βρε παιδιά, είπε, μη μαλώνετε. Χρειάζεται μια παστρικιά σε κάθε χωριό,
να μη βρίσκουν τον μπελά τους οι τίμιες. Είναι σαν τη βρύση του δρόμου,
μαθές· περνούν και πίνουν οι διψασμένοι· αλλιώς θα χτυπούσαν αράδα τις
πόρτες μας. Κι οι γυναίκες, όταν τους ζητήσεις νερό...

Στράφηκε, είδε το δάσκαλο:

--- Χατζη-Νικολή μου, του κάνει, ακόμα είσαι εδώ; Προεστός είσαι και του
λόγου σου, έχουμε σύναξη. Σκολειό έκαμες και τον καφενέ, σκόλασε κι έλα!

--- Να~’ρθω κι εγώ; είπε ο γερο-Χριστοφής κι έπαιξε το μάτι στην παρέα· κάνω
για Ιούδας.

Μα ο καπετάν Φουρτούνας είχε πάρει την ανηφοριά, βαριακουμπώντας το ραβδί
του στο καλντιρίμι. Δεν ήταν στα καλά του σήμερα· τον έσφαζαν πάλι οι
ρεματισμοί, μάτι δεν είχε κλείσει όλη νύχτα. Ήπιε πρωί πρωί δυο τρία
νεροπότηρα ρακή, για γιατρικό, μα του κάκου· οι πόνοι δε σκολνούσαν. Μήτε η
ρακή τους έβανε κάτω.

--- Μωρέ, αν δε ντρέπουμουν, μουρμούρισε, κι άρχιζα τις φωνές, μπορεί
ν’~αλάφρωναν οι πόνοι· μα έλα που δε με αφήνει το παντέρμο το φιλότιμο! Και
πρέπει να περπατώ ντούρος και να κάνω πως γελώ. Κι αν πέσει χάμω το ραβδί
μου, να μην αφήσω κανέναν κερατά να με βοηθήσει, μα να σκύψω μοναχός μου να
το πιάσω. Δάγκανε, μωρέ καπετάν Φουρτούνα, τα χείλια, όρτσα τα πανιά,
κατακέφαλα στο κύμα, βάρδα μην ντροπιαστείς! Μπόρα είναι, μαθές, κι η ζωή,
θα περάσει!

Έγρουζε και σιγοβλαστημούσε σκαμπανεβαίνοντας. Στάθηκε μια στιγμή, κοίταξε
γύρα, κανένας δεν τον έβλεπε· αναστέναξε κι αλάφρωσε λίγο. Κοίταξε κατά
πάνω, είδε στην κορυφή του χωριού ν’~ασπρογυαλίζει ανάμεσα από τα δέντρα το
σπίτι του παπά με τα λουλακιά παραθυρόφυλλα.

--- Πήγε κι έχτισε στην κορυφή του χωριού ο διαολόπαπας! μουρμούρισε· την
κατάρα του να~’χω!

Και πήρε πάλι τον ανήφορο.

Στο σπίτι του παπά είχαν κιόλα φτάσει δυο από τους προεστούς και κάθουνταν
σταυροπόδι στο μεντέρι, αμίλητοι, και περίμεναν τα τραταρίσματα. Ο παπάς
είχε μπει στην κουζίνα να δώσει διαταγές, κι η μοναχοκόρη του η Μαριορή
ετοίμαζε το δίσκο με τον καφέ, το δροσερό νερό και το γλυκό του κουταλιού.

Πλάι στο παραθύρι είχε θρονιαστεί ο πρώτος δημογέροντας της Λυκόβρυσης, από
μεγάλο τζάκι, αρχοντάνθρωπος, καλοθρεμμένος, με τσόχινο σαλβάρι, με
χρυσοκέντητο μεϊντάνι και μ’~ένα χοντρό χρυσό δαχτυλίδι στο δείχτη του
χεριού του --- η βούλα του με τα δυο κεφαλαία σφιχταγκαλιασμένα γράμματα:
Γ. Π., Γεώργιος Πατριαρχέας. Τα χέρια του ήταν παχιά και μαλακά, σα
Δεσπότη. Δε δούλεψε ποτέ του, είχε ένα τσούρμο φαμέγιους και κολίγους που
δούλευαν και τον τάιζαν. Και χόντρυνε τ’~άντερό του, φάρδυναν τα καπούλια
του, είχαν γίνει σαν της φοράδας, κρεμάστηκαν οι κοιλιές του, και τα
προγούλια του κατέβαιναν τρεις πατωσιές κι αναπαύουνταν, το ένα απάνω στο
άλλο, στο μαλλιαρό αφράτο απανωστήθι. Δυο τρία δόντια μπροστινά του~’λειπαν,
άλλο κουσούρι δεν είχε, κι όταν μιλούσε, φαφλάτιζε και μπερδεύουνταν· μα και
το κουσούρι αυτό πλήθαινε την αρχοντιά του, γιατί σε ανάγκαζε, όταν μιλούσε,
να σκύβεις για να καταλάβεις τι λέει.

Δεξά του, στη γωνιά, αδύναμος, λιγδοτάμπαρος, φτενοκέφαλος, με τσιμπλιασμένα
μάτια, με δυο χοντρές χερούκλες γεμάτες ρόζους, κάθουνταν συμμαζεμένος και
ταπεινός ο δεύτερος προεστός, ο πιο βαρβάτος νοικοκύρης του χωριού, ο
γερο-Λαδάς. Εβδομήντα χρόνια σκύβει στη γης, τη σκάβει, τη σπέρνει, τη
θερίζει, της φυτεύει ελιές κι αμπέλια, τη στύβει και της πίνει το αίμα. Ποτέ
του, από μικρό κουτσούβελο, δεν ξεκόλλησε από τη γης. Αχόρταγος, λιμασμένος,
έπεφτε απάνω της, της έδινε ένα και της ζητούσε χίλια, και ποτέ του δεν
έλεγε: «Δόξα σοι ο Θεός», μα πάντα μουρμούριζε, αφχαρίστητος. Και στα
γερατιά του δεν τον έφτανε πια η γης· όσο ζύγωνε στο θάνατο κι ένιωθε πως
λίγα πια ήταν τα καρβέλια του, βιάζουνταν να προλάβει και να φάει τον
κόσμο. Κίνησε λοιπόν να δανείζει με βαρύ διάφορο τους χωριανούς του· έβανα
οι δύστυχοι αμανάτι τ’~αμπελοχώραφά τους και τα σπίτια, έρχουνταν ο καιρός
της πληρωμής, δεν είχαν να πλερώσουν, έβγαιναν τα πράματα στο σφυρί, και
τα’~χαφτε ο γερο-Λαδάς.

Κι όλο κλαίγουνταν, κι όλο πεινούσε, κι η γυναίκα του γύριζε ξυπόλυτη, και
μια θυγατέρα, που~’χε καταφέρει να κάμει, την άφησε κι αυτή να πεθάνει,
γιατί δεν κάλεσε, όταν έπεσε στο κρεβάτι, γιατρό να τη δει.

«Πολλά έξοδα, είπε, οι μεγάλες πολιτείες είναι μακριά, πού να φέρνεις
γιατρό! Κι ύστερα, τι ξέρουν κι αυτοί; τον κακό τους τον καιρό! Έχουμε εδώ
τον παπά μας, αυτός κατέχει παλιά γιατροσόφια, θα τον πλερώσω να κάμει κι
ευκέλαιο, και θα γιάνει, και θα κοστίσει και πιο φτηνά.»

Μα τα ματζούνια του παπά πήγαν χαμένα, το ευκέλαιο δεν έπιασε, κι η κοπέλα
πέθανε, δεκαφτά χρονών, και γλίτωσε από τον κύρη της· γλίτωσε κι αυτός από
τα πολλά τα έξοδα του γάμου. Μια μέρα, λίγους μήνες μετά το θάνατό της,
κάθισε και τα λογάριασε: Προίκα τόσο απάνω κάτω, ρουχισμός, τραπέζια,
καρέκλες, τόσα, θ’~αναγκάζουνταν να καλέσει στο γάμο τους συγγενείς, κι
αυτοί θένε να φάνε τον περίδρομο, βάλε κρέατα, ψωμιά, κρασιά, τόσα... Έκαμε
τη σούμα, πάρα πολλά έξοδα, θα τον ξετίναζε η θυγατέρα του, δεν πειράζει το
λοιπόν, όλοι θα πεθάνουμε... Γλίτωσε κι από τα βάσανα του κόσμου – άντρες,
παιδιά, αρρώστιες, μπουγάδες. Τυχερή στάθηκε, ο Θεός σχωρέσ’~τη!


\section{Θεωρία πεδίων}

Μπήκε η Μαριορή με το δίσκο, χαιρέτησε χαμοβλεπούσα τους προεστούς, στάθηκε
μπροστά από τον άρχοντα. Χλωμή, μεγαλομάτα, γαϊτανοφρύδα, με δυο χάντρες
πλεξούδες καστανά μαλλιά, τυλιμένες στεφανωτά γύρα στο κεφάλι. Γέμισε ο
γερο-άρχοντας τρουλωτά το κουταλάκι του γλυκό βύσσινο, κοίταξε την κοπέλα,
σήκωσε το ποτήρι.

--- Στις χαρές σου, Μαριορή μου, ευκήθηκε· ο γιος μου βιάζεται.

Ήταν αρραβωνιασμένη η παπαδοπούλα με τον μοναχογιό του το Μιχελή, και ο
παπάς καμάρωνε που θα’~κανε τέτοιο συμπεθεριό και θα’~πιανε γρήγορα αγγόνια.

--- Δεν μπορώ να καταλάβω γιατί βιάζεται τόσο ο βλογημένος· δεν μπορεί,
λέει, πια... πρόσθεσε γελώντας ο άρχοντας κι έκλεισε το μάτι στην κοπέλα.

Κι αυτή κοκκίνησε ως το λαιμό, πλαντούσε, δεν μπόρεσε να μιλήσει.

--- Χαρές να’~χουμε! είπε ο παπα-Γρηγόρης, μπαίνοντας με μια μπουκάλα
μοσχάτο κρασί. Με την ευκή του Χριστού και της Παναγιάς!

Άγριος, κοτσονάτος, με διχαλωτή γενειάδα κάτασπρη, καλοθρεμμένος, μύριζε
λιβάνι και βούτυρο. Είδε την κόρη του να κοκκινίζει και, για ν’~αλλάξει
κουβέντα:

--- Πότε με το καλό θα παντρέψεις και την παρακόρη σου το Λενιό; ρώτησε.

Το Λενιό ήταν μια από τις μπασταρδοπούλες που ο άρχοντας είχε σκαρώσει με
τις παραδουλεύτρες του. Την είχε αρραβωνιάσει με τον ήμερο πιστό τσοπάνη
του, το Μανολιό, και την είχε αρχοντικά προικίσει μ’~ένα κοπάδι πρόβατα, που
τα’~βοσκε ο Μανολιός στο αντικρινό βουνό της Παναγιάς.

--- Αν θέλει ο Θεός, αποκρίθηκε, αυτές τις μέρες· το Λενιό, λέει,
βιάζεται. Βιάζεται το καλορίζικο· σηκώθηκε, μαθές, το βυζί του και θέλει να
βυζάξει γιο. «Μάης μπαίνει, μου’~πε προχτές, Μάης μπαίνει, αφεντικό, και
πρέπει να βιαστούμε.»

Γέλασε πάλι καλόκαρδα, τα προγούλια του σείστηκαν.

--- Το Μάη είπε, παντρεύονται τα γαϊδούρια, έχει δίκιο το Λενιό· πρέπει να
βιαστούμε. Άνθρωποι είναι κι αυτοί, και ας είναι και φαμέγοι.

--- Καλός είναι ο Μανολιός, έκαμε ο παπάς· καλά θα ζήσουν.


\englishtext

\chapter{Introduction}

Sitting on his balcony above the village square, the Agha of Lycovrissi
smoked his pipe and sipped raki. A thin, warm rain fell softly; on his fat
mustaches, which were freshly dyed black, tiny drops hung and twinkled;
warmed by the raki, the Agha licked his lips and enjoyed the cool. At his
right hand, holding a trumpet, stood Hussein, his squire and bodyguard, a
giant Oriental, wicked as a monkey, and with a squint. At his left, seated
on a velvet cushion with his legs tucked under him, a dimpled boy
ceaselessly relit his pipe for him and refilled his cup with raki.

The Agha half shut his heavy eyelids and savored this world below. All that
the good God has made is perfect, he thought: this world’s a real
success. Are you hungry? here’s bread and minced meat or pilaff with
cinnamon. Are you thirsty? here’s that water of youthfulness, raki. Are you
sleepy? God has made sleep; nothing like it when you’re sleepy. If you’re
angry, He’s made the whip and the raia’s buttocks. If you’re depressed, He’s
made the amanés, If, lastly, you want to forget all the sorrows and worries
of this world, He’s made Youssoufaki.

A wonderful artist, Allah! he murmured sentimentally, yes, indeed, a
wonderful artist Who knows His business and is ingenious, too. How the devil
did He get the idea of making raki and Youssoufaki?

The Agha’s eyes moistened with tears; he had drunk so much raki that his
soul was filled with tenderness. Leaning over his balcony, he watched the
raias strolling in the square, just shaved, in their best clothes, with
their broad red sashes, freshly washed breeches, long blue leggings. Some
wore the fez, others the turban, others the sheepskin cap. The smartest had
a sprig of basil or a cigarette behind the ear.


\section{The Goal of this Project}

It was Easter Tuesday; Mass was just over. Exquisite weather, tender: spring
sun and rain; the lemon blossoms were fragrant, the trees budding, the grass
reviving, Christ rising from every clod. The Christians were coming and
going across the square and embracing one another with the Paschal greeting:
Christ is risen! Risen indeed! --- after which they would go and sit at
Kostandis’s café or in the middle of the square under the old plane
tree. They ordered narghiles, with their long tubes and bubbling water, and
coffee, and at once there began an endless chatter, like the light rain.

This is what it’ll be like in Paradise, hazarded Charaiambos the beadle,
soft sunshine, a gentle rain falling without a sound, lemon trees in
blossom, narghiles, and agreeable conversation, forever and ever.

At the other end of the square, behind the plane tree, freshly whitewashed
and with its graceful bell tower beside it, rose the village church: the
Crucifixion. Today its doorway was decked with palm and laurel branches. All
around were small shops and stalls:

There was Panayotaros the saddler’s (a clown with the nickname
Plaster-eater: once, when a plaster statuette of Napoleon was brought to the
village, he had bolted it. After that, they had brought another, of Kemal
Pasha; bolted again. Finally, one of Venizelos; bolted like the others).

Next door, Andonis the barber’s with its signboard saying Andonis. Above the
door an inscription in thick, dark-red letters announced: Teeth also
extracted.

Farther on, the butcher’s shop kept by old lame Dimitri: Fresh
calves’~heads, Herodiade. Every Saturday he killed a calf; before doing so
he gilded its horns, painted its forehead, hung red ribbons round its neck
and, limping, led it through the village, singing its virtues.

Lastly the famous Café Kostandis, a long narrow room where it was cool and
there was always a balmy smell of coffee and tobacco and, in winter,
sage. On its walls hung --- the pride of the village --- three impressive
portraits on glossy paper: on one side Saint Genevieve, half naked in a
tropical forest; on the other, Queen Victoria, with blue eyes and an
enormous nurse’s bosom; right in the middle, hard-faced, grey-eyed,
glowering and wearing a tall Astrakhan cap, Kemal Pasha.

Fine men, all these villagers, hard workers, good fathers; and the Agha,
too, was a fine man with his love for raki, for the heavy scents --- musk
and patchouli --- and for the pretty boy seated at his left on the velvet
cushion. Amused, the Agha gazed upon the Christians, like a shepherd upon
his flock, and was well pleased.

Excellent fellows, he thought, this year again they’ve filled my cellar with
their Easter presents --- cheeses, coronets of sesame bread, brioches,
scarlet eggs... One of them, may Heaven preserve him, has brought me a box
of Chian mastic for my Youssoufaki to munch and make his little mouth smell
nice.


\section{The C++ Programming Language}

The Agha felt happy: My cellar, he thought, is bursting with good things,
the rain is falling gently, the cocks are crowing, and, close beside me,
crouched at my feet, my Youssoufaki munches his mastic and smacks his
tongue. The Agha suddenly felt his heart overflowing. He bent his neck and
was about to begin singing the amanés; but the effort was too great; turning
toward Hussein he signed to him to put his trumpet to his lips to silence
the raias. After which, he turned to his left:

``Sing, Youssoufaki (my blessing be upon you), sing me `Dounia tabir, rouya
tabir, aman, aman!' Sing it to me or I shall burst!''

The pretty boy, without hurrying, took the mastic from his mouth, stuck it
on his bare knee, leaned his right palm against his cheek and began to sing
his Agha’s favorite amanés: ``World and dream are but one, aman, aman!''

His fluting voice went up and down with dove-like cooings. The Agha,
enchanted, shut his eyes and, as long as the boy sang, forgot to drink.

One of his good days, whispered Kostandis as he served coffee, blessed be
raki!

Blessed be Youssoufaki, said Yannakos, smiling maliciously. He was the
village carrier and courier: a thick pepper-and-salt beard, eyes of a bird
of prey.

``A curse on destiny, the blind hag, for having made him an Agha and us
raias,'' growled the priest’s brother, Hadji Nikolis, the village
schoolmaster; a dry individual with glasses and a jutting Adam’s apple that
bobbed up and down when he talked.

\begin{figure}[t]
\setlength\partopsep{-\topsep}% adjusts vertical space after the listing
\begin{minted}[frame=lines,linenos,numbersep=5pt]{c++}
#include <iostream>
int main() {
  std::cout << "Hello world!" << std::endl;
}
\end{minted}
\caption{Filled with joy, the Agha decides to write some code.%
  \label{fig:hello-english}}
\end{figure}


\section{The Structure of this Thesis}

He kindled, thought of his ancestors, sighed:

There was a time when our people, the Hellenes, were the masters of these
lands. The wheel turned and the Byzantines came --- Hellenes too, and
Christians. The wheel turned again, and the children of Hagar came... But
Christ rose again, my friends, our country will rise again, too! Here,
Kostandis, another round!

The amanés finished, the exquisite boy put back the mastic into his mouth
and resumed his sleepy rumination. The trumpet sounded again: the raias
could now laugh and shout freely.


\chapter{Theoretical Background}

Captain Fortounas, one of the five village Elders, appeared at the door of
the café: a tall, corpulent character, formerly a boat owner who for many
years had ploughed the Black Sea, transporting Russian corn and not above
smuggling. He had not a hair on his chin: complexion olive, a parchment
skin, deep wrinkles, tiny, sparkling, jet-black eyes. He had grown old, and
his boat with him. One night it had smashed on a reef off Trebizond:
shipwrecked and disillusioned, Captain Fortounas had come back to his native
village, intending to put away as much raki as possible and, when the time
came, to turn his face to the wall and die. His eyes had seen too many
things; he had had enough; no, he hadn’t had enough; he was tired, but
ashamed to admit it.


\section{Programming Language Semantics}

Today he was wearing his captain’s boots, his yellow belt and his notable’s
cap --- real Astrakhan; in his hand the long staff of an Elder. Two or three
villagers stood up respectfully and invited him to a glass of raki.

No time, my children, even for raki, said he; Christ is risen! I’m going to
the priest’s house, we’ve a notables’~meeting. In less than an hour’s time,
all those who’ve been invited should be there. Quick, cross yourselves and
come; you know what the business is today, surely. Ah, one of you should go
and fetch saddler Panayotaros with his devil’s beard; we need him badly.

He was silent for a moment, blinking, then said maliciously:

If he isn’t at home, he’ll be at the widow’s. They all burst out laughing.

Christofis, the old muleteer, who had learned love in his young days --- and
a heavy price he had paid for it --- came out with this violent retort:

What are you sniggering at, you chicken heads? He’s quite right. Do as you
damn well like, Panayotaros, and pay no attention to what they say! Life’s
short, death’s long. Go ahead, my lad!

Fat Dimitri, the butcher, shook his close-shaved skull:

God preserve the widow, our Katerina! The devil knows how many horns she
saves us from!

Captain Fortounas laughed.

There, children, don’t argue. Every village should have its odd woman, so
that the honest ones mayn’t be upset. It’s like the fountain by the
roadside, that’s it: those who’re thirsty stop there and have a
drink. Otherwise they come knocking at all our doors, one after the other;
and the women, when they’re asked for water...

He turned, and noticed the schoolmaster.

What, you here still, old one? Aren’t you on the council, too? Turn even the
café into a school? Class is over, come along!

Don’t you want me to come, too? said old Christofis, with a wink to the
company. I’d do for Judas.

But Captain Fortounas had already started up the slope, leaning heavily on
his stick. He was not in good shape today; his rheumatism was at him with
its pincers; he had not closed an eye all night. Of course he had gulped
down two or three big glasses of raki that morning by way of remedy, but,
devil take it, the pain hadn’t given him a moment’s peace. Even the raki
hadn’t done the trick.

If I weren’t ashamed, I’d start screaming: that might quiet the pain a
bit. But there’s this damned self-respect and look jovial. And if I let slip
my stick, I shan’t let any young imp help me, I shall stoop and pick it up
by myself... Come on, Captain Fortounas, bite your lips, hoist your sails,
steer into the waves, my lad! Don’t go covering yourself with shame! By God,
sir, life, too, is a squall; it’ll pass!

He growled and blasphemed to himself. As he climbed the hill a lurch hurled
him from one wall to the other. He stopped a moment and looked around him:
nobody was looking; he sighed noisily and this relieved him a
little. Raising his eyes toward the upper end of the village, he caught
sight of a white patch among the trees, the indigo-shuttered house of the
priest.

What the devil was he thinking of, the old sod, going and building his house
at the top of the hill, he grumbled. His curse upon me! And he resumed his
climb.

Two notables had already arrived; seated cross-legged on the divan they were
waiting in silence for the dish to be brought in. The priest had gone to
give his orders in the kitchen, where his only daughter, Mariori, was
preparing the coffee, the cool water, and the preserves.

Close to the windows the first Elder of Lycovrissi sat throned: corpulent,
lordly in manner, wearing breeches of fine linen, a gold-braided bolero and
a stout gold ring on his forefinger --- his seal with his initials
interlaced: G.P., George Patriarcheas. His hands were fat and soft, like a
bishop’s. He had never done anything in his life, having a whole tribe of
servants and serfs working for his service. He had swollen intestines,
spreading buttocks, a pendulous paunch and three tiers of chin which, one on
top of another, rested on a well-fleshed, hairy chest. He had two or three
teeth missing in front --- it was his only defect and made him lisp and
stammer. But even this defect added to his distinction, for it forced
whoever was speaking with him to lean toward him in order to hear what he
said.

At his right in a corner, thin, grubby, with a cadaverous head, bleary eyes
and huge callused hands, sat in a heap, humble and self-effacing, the second
notable: the richest man of the village, old Ladas. Bent over the land, for
seventy years he had ploughed it, sowed it, reaped it, planted it with
olives and vines, pressed it, drunk its blood. Not once since he was a lad
had he cleaned it off him. Insatiable, he had hurled himself on it,
demanding that it yield him a thousand to one. Yet he never once said, God
be praised! but grumbled, eternally discontented. Now, in his old age, the
land was not enough for him. As his death approached and he felt himself
getting near the end of his scroll, he was impatient to devour the whole
village. He had taken to lending money at high rates; men down on their luck
pledged their vineyards and their houses and, when payment was due and they
hadn’t a penny, saw their property sold at auction and old Ladas gobbling up
the lot.

And yet he moaned without stopping, and never had enough to eat; Penelope,
his wife, went barefoot, and when the one and only daughter he had managed
to produce fell ill, he had let her die through not sending for the doctor.

It costs a lot of money, he had said; the big towns are a long way off; how
can I get a doctor out? And then, what do they know, more than others?
Plague on them! We’ve got our priest; he knows the old medicines, and I
shall only pay him for the extreme unction. She’ll get well just the same,
and it’ll be cheaper.

But the priest’s electuaries were no use, the holy oils had no effect, and
the young girl died at seventeen and shook free of her father. He, too, was
freed from the expense of a wedding: one day, not long after his daughter’s
death, he had done the accounts: dowry about so much; linen, tables, chairs
so much. And then wouldn’t he have been obliged to invite to the wedding all
those relatives, who put away victuals like the gluttons they are? So then
meat, bread, wine so much... He had added it all up, a pretty figure: his
daughter would have reduced him to beggary. So what did it matter? We shall
all die... Besides, she had escaped from the worries of this world ---
husband, children, illness, housework... In fact, she’d been lucky, God rest
her soul!


\section{Domain Theory}

Mariori came in with the dish, greeted the notables and, with eyes cast
down, stopped first in front of the archon. She was pale and had big eyes,
fine-drawn eyebrows, and two great tresses of chestnut hair made her a
crown. The old archon helped himself to a full spoonful of wild bitter
cherry conserve, looked at the young girl and raised his glass.

To your loves, Mariori; my son’s getting impatient.

The priest’s daughter was engaged to his only son, Michelis, and the priest
boasted that such a match would soon bring him grandchildren.

I begin to see why he’s so impatient, the young dog. He can’t stand it any
more, he says, the old man added with a laugh and winked at the young girl.

She blushed to the roots of her hair and became tongue-tied.

Joy to us all! said Grigoris, the priest, as he brought in a bottle of
muscatel wine. With the blessing of Christ and the Virgin!

Green still, getting stout, with a perfectly white forked beard, he smelled
of incense and fat. He saw his daughter’s confusion and, to change the
subject:

And when, God willing, he asked, do you think of marrying off that adopted
daughter of yours as well --- Lenio?

Lenio was one of the bastards the archon had begotten on his servants. He
had betrothed her to his shepherd, the faithful Manolios, and given her a
generous dowry in the shape of a flock of sheep, which Manolios minded on
the Mount of the Virgin, close to the village.

God willing, quite soon, he replied; Lenio’s in a hurry. She’s in a hurry,
the lucky girl! I do believe her breasts are swelling and she needs to nurse
a son. `Here’s the month of May,' she said to me the other day, `here’s the
month of May, master, it’s high time.'

He laughed again, wholeheartedly, and his triple chin shook.

Only donkeys wed in May, he said; she’s right, Lenio is, it’s high
time. They’re human, too, after all, even if they are servants.

Manolios is a good lad, said the priest; they will be happy.

\selectlanguage{greek}


%%%  Bibliography

% You shouldn't want to include all the contents of thesis.bib
% in your bibliography (do you?)
\nocite{*}

\bibliographystyle{softlab-thesis}
\bibliography{thesis}


%%%  Appendices

\backmatter

\appendix

\chapter{Ευρετήριο συμβολισμών}

$A \rightarrow B$ : συνάρτηση από το πεδίο $A$ στο πεδίο $B$.

\chapter{Ευρετήριο γλωσσών}

\begin{description}
\item[C++:] πώς θα βγάλω λεφτά...
\item[Haskell:] η γλώσσα της ζωής μου αλλά πάνε οι μπύρες...
\item[Javascript:] χα, χα, χα...
\item[Python:] πώς θα τελειώνω για να πάω για μπύρες...
\end{description}


\chapter{Ευρετήριο αριθμών}

\begin{description}
\item[17:] ask Zachos.
\item[42:] life, the universe and everything --- ask Douglas.
\end{description}


%%%  End of document

\end{document}
