\documentclass[diploma]{softlab-thesis}

%%%
%%%  Add and configure the packages that you need for your thesis
%%%

\usepackage{minted}
\usepackage{amsmath}
\usepackage{enumitem}
\usepackage{bm}

\usepackage{tikz}
\usetikzlibrary{calc}
\usetikzlibrary{arrows.meta}
\usetikzlibrary{positioning}
%% I define a "style" for the "cells"
\tikzset{
  my cell/.style={
    draw,
    minimum size=5ex,
    node distance=0pt,
  }}
%% to make changing appearances for various aspects of
%% the diagram easier, I define several macros which need
%% only be changed here.  Even though two of these macros
%% have the same definition, I use two to allow the possibility
%% that these styles could be diffferent.
\newcommand\mynodestyle{\sffamily}
\newcommand\myrelocatestyle{\sffamily\bfseries}
\newcommand\mynumberstyle{\sffamily}

%%%
%%%  The document
%%%

\begin{document}

%%%  Title page

\frontmatter

\title{Σχεδίαση και Υλοποίηση μιας Καταπληκτικής Γλώσσας Προγραμματισμού}
\author{Νίκος Μαυρογεώργης}
\authoren{Nikos Mavrogeorgis}
\date{Αύγουστος 2019}
\datedefense{26}{8}{2019}

\supervisor{Νικόλαος Σ. Παπασπύρου}
\supervisorpos{Καθηγητής Ε.Μ.Π.}

\committeeone{Νικόλαος Σ. Παπασπύρου}
\committeeonepos{Καθηγητής Ε.Μ.Π.}
\committeetwo{Πέτρος Παπαδόπουλος}
\committeetwopos{Επίκ. Καθηγητής Ε.Μ.Π.}
\committeethree{Γεώργιος Νικολάου}
\committeethreepos{Αν. Καθηγητής Ε.Κ.Π.Α.}

\TRnumber{CSD-SW-TR-42-17}  % number-year, ask nickie for the number
\department{Τομέας Τεχνολογίας Πληροφορικής και Υπολογιστών}

\maketitle


%%%  Abstract, in Greek

\begin{abstractgr}%
  Σκοπός της παρούσας εργασίας είναι αφενός η σχεδίαση μίας απλής
  γλώσσας υψηλού επιπέδου με υποστήριξη για προγραμματισμό με
  αποδείξεις, αφετέρου η υλοποίηση ενός μεταγλωττιστή για τη γλώσσα
  αυτή που θα παράγει κώδικα για μία γλώσσα ενδιάμεσου επιπέδου
  κατάλληλη για δημιουργία πιστοποιημένων εκτελέσιμων.

\begin{keywordsgr}
  Γλώσσες προγραμματισμού,
  Προγραμματισμός με αποδείξεις,
  Ασφαλείς γλώσσες προγραμματισμού,
  Πιστοποιημένος κώδικας.
\end{keywordsgr}
\end{abstractgr}


%%%  Abstract, in English

\begin{abstracten}%
  The purpose of this diploma dissertation is on one hand the design
  of a simple high-level language that supports programming with
  proofs, and on the other hand the implementation of a compiler for
  this language. This compiler will produce code for an
  intermediate-level language suitable for creating certified
  binaries.

  \begin{keywordsen}
  Programming languages,
  Programming with proofs,
  Secure programming languages,
  Certified code.
\end{keywordsen}
\end{abstracten}


%%%  Acknowledgements

\begin{acknowledgementsgr}
  Ευχαριστώ θερμά τον επιβλέποντα καθηγητή αυτής της διατριβής,
  κ.~Νίκο Παπασπύρου
\end{acknowledgementsgr}

\begin{acknowledgementsen}
  I would like to thank all the people who supported my work and helped me get
  results of better quality.  
\end{acknowledgementsen}


%%%  Various tables

\tableofcontents
%\listoftables
\listoffigures
%\listofalgorithms

%%%  Main part of the book

\mainmatter

\chapter{ Εισαγωγή }

Κάθε γλώσσα προγραμματισμού από τη φάση σχεδιασμού της έχει ακριβείς κανόνες που καθορίζουν τη συντακτική δομή των σωστά διατυπωμένων προγραμμάτων της. \cite{dragon}
Στη C, για παράδειγμα, ένα πρόγραμμα αποτελείται από συναρτήσεις, μια συνάρτηση από δηλώσεις και εντολές, μια εντολή από εκφράσεις, και ούτω καθ' εξής.
Πρακτικά όλες οι γλώσσες που χρησιμοποιούνται συχνά σήμερα, τόσο φυσικές γλώσσες όσο και γλώσσες μηχανής, βασίζονται στην έκφραση της πληροφορίας με γραμμικό τρόπο, 
όπως οι ακολουθίες συμβόλων  \cite{ford-thesis}.
Το κείμενο σε μία γραπτή γλώσσα συνήθως αναπαρίσταται ως μία  \textit{ συμβολοσειρά}, δηλαδή μια ακολουθία χαρακτήρων που προέρχονται από ένα τυποποιημένο σύνολο. 
Το πρώτο πράγμα που πρέπει να κάνει οποιαδήποτε εφαρμογή επεξεργασίας γλώσσας, 
είναι να μετατρέψει αυτές τις συμβολοσειρές σε πιο αφηρημένες δομές όπως λέξεις, φράσεις, προτάσεις, εκφράσεις ή εντολές.
Η διαδικασία που εξάγει τέτοια χρήσιμη δομημένη πληροφορία από γραμμικό κείμενο είναι γνωστή ως συντακτική ανάλυση ή parsing.

\section{ Ορισμός συντακτικού}

Προκειμένου να κατασκευάσουμε έναν συντακτικό αναλυτή (ή  parser) για μία γλώσσα, ή ακόμα και για να ορίσουμε τυπικά ποια είδη συμβολοσειρών έχουν νόημα σε αυτήν, πρέπει να έχουμε έναν τρόπο να για να εκφράσουμε και να κατανοήσουμε τη συντακτική δομή της.
Για το σκοπό αυτό συνήθως χρησιμοποιούμε κάποια  \textit{ γραμματική}, που είναι μία συμπαγής αναπαράστησαση της δομής μίας γλωσσας, εκφρασμένη σε κάποια άλλη (ιδανικά μικρή και απλή) γλώσσα.
Η γλώσσα της οποίας της δομή προσπαθούμε να αναπαραστήσουμε είναι η  \textit{ γλώσσα αντικείμενο}, ενώ η γλώσσα στην οποία \textit{ εκφράζεται} η συντακτική δομή ονομάζεται  \textit{γραμματική ορισμού γλώσσας}. 

Ο πιο συνηθισμένος τύπος γραμματικής σήμερα, είναι οι \textit{γραμμτικές χωρίς συμφραζόμενα ( context-free grammars - CFG)}, εκφρασμένες σε  Backus-Naur Form (BNF). 
Μία γραμματική χωρίς συμφραζόμενα ουσιαστικά εκφράζει ένα σύνολο αμοιβαίως αναδρομικών κανόνων, οι οποίες περιγράφουν πώς μπορουυν να γραφτούν οι συμβολοσειρές που περιγράφονται στη γλώσσα.
Κάθε κανόνας ή \textit{ παραγωγή}  σε μία  CFG καθορίζει έναν τρόπο με τον οποίο μία συντακτική μεταβλητή ή  \textit{ μη τερματικό} μπορεί να αντικατασταθεί σε μία συμβολοσειρά. 
Ένα μη τερματικό μπορεί να αντικατασταθείξλ σε μία συμβολοσειρά που μπορεί να περιέχει άλλα μη τερματικά, τα οποία θα αντικατασταθούν με τη σειρά τους, ώσπου να μην υπάρχουν άλλα.
Επειδή υπάρχουν πολλοί τρόποι για να αντικατασταθεί ένα μη τερματικό, η γραμματική μπορεί να εκφράσει ένα άπειρο σύνολο καλώς ορισμένων συμβολοσειρών.
Η συντακτική ανάλυση μιας συμβολοσειράς, της οποίας η σύνταξη περιγράφεται από μία  CFG, περιλαμβάνει την ανάποδη διαδικασία: 
να καθοριστεί από μία πλήρως ανεπτυγμένη συμβολοσειρά, η οποία περιέχει μόνο ατομικούς χαρακτήρες ή  \textit{ τερματικά}, ποια ακολουθία (ή ακολουθίες βημάτων) αντικατάστασης, αν υπάρχουν, οδηγούν στην παραγωγή της συμβολοσειράς.
Αυτή η εργασία περιπλέκεται καθώς οι  CFGs  συνήθως περιέχουν αμφισημίες:
σε  \textit{τοπικό επίπεδο}, όπου η σωστή ερμηνεία ενός τμήματος της συμβολοσειράς μπορεί να καθοριστεί μόνο από τα συμφραζόμενα; 
σε \textit{καθολικό επίπεδο}, όπου ολόκληρη η συμβολοσειρά μπορεί να έχει πολλαπλές έγκυρες συντακτικές ερμηνείες.

\section{ Parsing Expression Grammars}
 Η θεωρία και η πράξη της συντακτικής ανάλυσης βασίζεται σε \textit{ παραγωγικά ( generative)} συστήματα, όπως οι κανονικές εκφράσεις και οι γραμματικές χωρίς συμφραζόμενα, στα οποία η γλώσσα ορίζεται τυπικά μέσα από κανόνες οι οποίοι όταν εφαρμοστούν αναδρομικά παράγουν συμβολοσειρές της γλώσσας.
Αντίθετα, σε ένα \textit{ αναγνωριστικό σύστημα ( recognition-based system)} η γλώσσα ορίζεται μέσα από κανόνες ή κατηγορήματα που αποφασίζουν εάν η δοθείσα συμβολοσειρά ανήκει στη γλώσσα \cite{ford-peg}.
Οι απλές γλώσσες μπορούν να εκφραστούν εξίσου εύκολα και στα δύο συστήματα.
 Για παράδειγμα, το $\{ s \in \mathbf{a}^* | s = {(\mathbf{a}\mathbf{a})}^n\}$ είναι ένας παραγωγικός ορισμός μιας γλώσσας με ένα μόνο γράμμα στο λεξιλόγιό της, της οποίας οι συμβολοσειρές 
 κατασκευάζονται συνενώνοντας ζεύγη από $ \mathbf{a}$.
 Από την άλλη, το $\{ s \in \mathbf{a}^* | (\lvert s \rvert mod2=0)\}$, είναι ένας αναγνωριστικός ορισμός, όπου μία συμβολοσειρά από  $\mathbf{a}$'s γίνεται αποδεκτή μόνο αν το μήκος της είναι άρτιο.
 
  Αν και το παραγωγικό μοντέλο χρησιμοποιείται ευρύτατα στη θεωρία γλωσσών, οι πιο πολλές πρακτικές γλωσσικές εφαρμογές περιλαμβάνουν την αναγνώριση και τη δομική ανάλυση συμβολοσειρών. 
Εμείς θα χρησιμοποιήσουμε το αναγνωριστικό μοντέλο για τη σύνταξη της γλώσσας, η οποία θα περιγράφεται από τις  \textit{Parsing Expression Grammars (PEGs)}. Αυτές μοιάζουν με τις γραμματικές χωρίς συμφραζόμενα με την προσθήκη χαρακτηριστικών από τις κανονικές εκφράσεις, όμοια με την  Extended BNF σημειογραφία.

\section{ Packrat Parsing}
Με βάση τις Parsing Expression Grammars και το αναγνωριστικό σχήμα, ο απλούστερος τρόπος να αναλυθεί (συντακτικά) μία συμβολοσειρά έιναι μέσω ενός αναδρομικού καθοδικού συντακτικού αναλυτή, με δυνατόττηα για οπισθαναχώρηση ( backtracking). 
Ωστόσο, σε πολλές  PEGs  ένας συνήθης καθοδικός αναλυτής μπορεί να κάνει εκθετικό χρόνο για να αναγνωρίσει μία συμβολοσειρά στην είσοδο. 
Και αυτό διότι η οπισθαναώρηση μπορεί να οδηγήσει σε πλεονάζοντες υπολογισμούς ενδιάμεσων αποτελεσμάτων.
Αν, όμως, χρησιμοποιηθεί \textit{ υπομνηματισμός ( memoisation)} για να αποφευχθούν τέτοιοι πλεονασμοί, μία  PEG  μπορεί να αναλυθεί σε γραμμικό χρόνο με το μήκος της εισόδου.
Ένας τέτοιος συντακτικός αναλυτής ονομάζεται \textit{Packrat Parser}.

\section{ Αυτόματη παραγωγή των  Packrat Parsers}
Αν και το  packrat parsing είναι εύκολο να υλοποιηθεί με το χέρι, θα ήταν ακόμη καλύτερο να μπορούσαμε να κατασκευάσουμε τέτοιους συντακτικούς αναλυτές με αυτόματο τρόπο, όμοια με το  YACC  στον κόσμο της C. 
Ένας \textit{γεννήτορας συντακτιών αναλυτών (parser generator)} παίρνει ως είσοδο μία τυπική περιγραφή της γραμματικής και "γεννάει" τον αντίστοιχο συντακτικό αναλυτή  σε  C++.
Η περιγραφή που δέχεται ο δικός μας γεννήτορας βασίζεται στη σημειογραφία των  PEGs. 
Τέλος, παρέχει και τη δυνατότητα να επιστραφεί το συντακτικό δέντρο μετά την ανάλυση, ενώ επεκτείνεται εύκολα ώστε να κρατήσει και άλλες σημασιολογικές τιμές.
Οι βασικές ιδέες για την υλοποίησή του πηγάζουν από πρότερη δουλειά που έχει γίνει σε  Java \cite{fowler} \cite{lgi}.

\section{  Παράλληλο Packrat Parsing}
Αφού  υλοποιήσουμε έναν  packrat parser generator σε C++, η βασική συνεισφορά που θέλουμε να κάνουμε στα πλαίσια της διπλωματικής, είναι να εξετάσουμε κατά πόσο και πώς ένας σειριακός packrat parser μπορεί να παραλληλοποιθεί ώστε να τρέχει αποδοτικότερα σε ένα πολυπύρηνο σύστημα με αρχιτεκτονική κοινής μνήμης.
Η πρώτη προσέγγιση που έχει δοκιμαστεί \cite{fowler} είναι να χωρίζεται η είσοδος σε κομμάτια και να αναλαμβάνει ένα νήμα να υπολογίσει τα κελιά του συντακτικού αναλυτή που αφορούν το συγκεκριμένο κομμάτι. (% TODO: add more about seith fowler)
H  πρώτη δική μας προσέγγιση ήταν να αντί για υπομνηματισμό, να χρησιμοποιήσουμε δυναμικό προγραμματισμό, τον οποίο επειτα δοκιμάσαμε να παραλληλοποιήσουμε.
Ακολούθως, εφαρμόσαμε παραλληλοποίηση στην πράξη της \textit{ επιλογής ( choice)} των  PEGs % TODO: add more about pht
Τέλος, αφού υλοποιήσαμε το  Elastic % TODO: 

\section{  Η δομή της εργασίας}

\chapter{ Parsing Expression Grammars }

Οι δύο πιο συνηθισμένες μέθοδοι για να περιγραφεί η σύνταξη μίας γλώσσας σήμερα είναι οι κανονικές εκφράσεις και οι γραμματικές χωρίς συμφραζόμενα. 
Αυτοί οι φορμαλισμοί, ωστόσο, δεν είναι σε καμία περίπτωση ο μοναδικός τρόπος ορισμούς της συντακτικής δομής μίας γλώσσας. 
Ένα ακόμη χρήσιμο πρότυπο περιγραφής της σύνταξης είναι οι  \textit{Parsing Expression Grammars (PEGs)} \cite{ford-peg}, οι οποίες μοιάζουν με τις γραμματικές χωρίς συμφραζόμενα, αλλά έχουν και ορισμένες θεμελιώδεις διαφορές. 
Δαισθητικά, μια γραμματική χωρίς συμφραζόμενα μας περιγράφει το πώς  \textit{κατασκευάζεται} μία συμβολοσειρά που ανήκει σε κάποια γλώσσα, ενώ οι  PEGs  το πώς  \textit{αναλύεται} η σημβολοσειρά ώστε να προκύψει δομμική πληροφορία για αυτή - εξ ου και το όνομα  "parsing language". 

 Δεδομένου ότι η περιγραφή των γλωσσών (συνήθως) γράφεται από ανθρώπους και διαβάζεται από μηχανές, οι  PEGs αποτελούν διαισθητικά ένα πιο κατάλληλο εργαλείο προσδιορισμού 
από της γραμματικές χωρίς συμφραζόμενα. 
Ο σχεδιαστής της γραμματικής είναι ευκολότερο να σκέφτεται πώς αναλύεται μία δοσμένη συμβολοσειρά στα συστατικά της, παρά πώς θα γεννηθεί ( generate) η συμβολοσειρά μέσα από τους κανόνες της γραμματικής.
% TODO: example to prove aforementioned claim

Πολλά συντακτικά ιδιώματα των σύγχρονων γλωσσών προγραμματισμού εκφράζονται ευκολότερα και πιο "φυσικά" σε  Parsing Expression Grammars. 
Επιπρόσθετα, οι PEGs μπορούν να αναλυθούν συντακτικά σε γραμμικό χρόνο, χρησιμοποιώντας το  Packrat Parsing  που περιγράφεται σε επόμενη ενότητα, ενώ μόνο μία συγκεκριμένη υποκλάση των γλωσσών χωρίς συμφραζόμενα μπορεί να αναλυθεί σε γραμμμικό χρόνο.

\section{Ορισμός των  Parsing Expression Grammars}
 Όπως με τις  Context Free,  οι  Parsing Expression Grammars χρησιμοποιούν τόσο τερματικά όσο και μη τερματικά σύμβολα και αποτελούνται από ένα σύνολο κανόνων που παρέχουν ορισμούς για τα μη τερματικά.
Κάθε κανόνας μπορεί να αναφέρεται σε άλλους κανόννες της γραμματικής αναδρομικά. Θα ακολουθήσουμε το συμβολισμό `$ n \leftarrow e $', όπου το $ n$ είναι ένα μη τερματικό και το $e$ είναι μία έκφραση που θα οριστεί ακολούθως. 
Η χρήση του αριστερού βέλους αντί του δεξιού εκφράζει την διασθητική διαφορά στην "ροή της πληροφορίας" που διακρίνει τις  PEGs από τις CFGs. 
Ενώ, οι κανόνες των CFG εκφράζουν "παραγωγές" από μη τερματικά στις αντίστοιχες εκφράσεις τους, οι κανόνες των  PEGs αναπαριστούν "αφαιρέσεις" από τις εκφράσεις στους αντίστοιχους κανόνες. 
Επιπλέον, οι παραγωγές εκφρασμένες σε  CFG αναπαριστούν πράξεις σε ολόκληρες συμβολοσειρές, ενώ οι αφαιρέσεις σε μία  PEG  αναπαριστά πράξεις σε προθέμνατα της συμβολεοσειράς στην είσοδο.

Σύμφωνα με τον συμβολισμό των  Parsing Expression Grammars, οι εκφράσεις σχηματίζονται ως εξής:

 \begin{description}[font=$\bullet$\scshape\bfseries]

   \item[ Κενή συμβολοσειρά:] 
	 `()' είναι μία έκφραση που υποδηλώνει την άδεια συμβολοσειρά. 
	 Η ερμηνεία της είναι "Μην προσπαθήσεις να διαβάσεις τίποτα: απλά επίστρεψε επιτυχώς χωρίς να καταναλώσεις τίποτα από την είσοδο."

   \item[ Τερματικό:] 
	 Αν το $ \alpha $ είναι ένα τερματικό σύμβολο (π.χ. ένας χαρακτήρας μόνος του), τότε το `$ \alpha$' είναι μία έκφραση της οποίας η ερμηνεία είναι: 
	 "Αν το επόμενο τερματικό στην είσοδο είναι $ \alpha $ τότε κατανάλωσε ένα τερματικό και επίστρεψε επιτυχώς. αλλιώς, απότυχε και μην καταναλώσεις τίποτα."

   \item[ Μη Τερματικό:]
	 Αν το $ A $ είναι ένα μη τερματικό σύμβολο , τότε το  `$ A $' είναι μία έκφραση της οποίας η ερμηνεία είναι:
	 "Προσπάθησε να διαβάσεις την είσοδο με βάση τον κανόνα που αντιστοιχεί στο  $ A $ και επίστρεψε επιτυχώς ή απότυχε αντίστοιχα."

   \item[ Ακολουθία:]
	 Αν $ e_1, e_2, \ldots, e_n $ είναι εκφράσεις, τότε το  `$(e_1 e_2 \ldots e_n)$' είναι μία έκφραση της οποίας η ερμηνεία είναι: 
	 "Πρώτα προσπάθησε να διαβάσεις μία συμβολοσειρά ώστε να επιτύχει η $e_1$. 
	 Αν η $ e_1$ επιτύχει, τότε προσπάθησε να διαβάσεις μία συμβολοσειρά ώστε να επιτύχει η  $e_2$, ξεκινώντας από το σημείο της εισόδο που δεν κατανάλωσε η  $e_1$. 
	 Αν η  $e_2$ επιτύχει τότε συνέχισε με  την  $e_3$ κ.ό.κ μέχρι την  $e_n$.
	 Αν και οι  $n$ εκφράσεις αναγνωριστούν επιτυχώς διαδοχικά, τότε επίστρεψε επιτυχώςκαι κατανάλωσε όλα τα αντίστοιχα κομμάτια της εισόδου.
	 Αν οποιαδήποτε υποέκφραση αποτύχει, τότε όλη η ακολουθία αποτυγχάνει συνολικά χωρίς να καταναλώνει τίποτα από την είσοδο."

   \item[ Διατεταγμένη Επιλογή] Αν $ e_1, e_2, \ldots, e_n $ είναι εκφράσεις, τότε το `$(e_1 / e_2 / \ldots / e_n)$' είναι μία έκφραση της οποίας η ερμηνεία είναι η εξής: 
	 "Πρώτα προσπάθησε να διαβάσεις μία συμβολοσειρά ώστε να επιτύχει η $e_1$. 
	 Αν αυτό πετύχει τότε η επιλογή επιστρέφει επιτυχώς καταναλώνοντας το αντίστοιχο κομμάτι της εισόδου.
	 Αλλιώς, προσπάθησε με την $e_2$ και την αρχική είσοδο, μετά με την $e_3$, κ.ό.κ, μέχρις ότου να φτάσεις στην $e_n$, σταματώντας στην πρώτη εναλλακτική που θα επιτύχει.
	 Αν καμία από τις $n$ εναλλακτικές δεν πετύχουν, τότε απότυχε χωρίς να καταναλώσεις τίποτα από την είσοδο."
	 Η έκφραση `$(e_1 / e_2)$' μπορεί να διαβαστεί ως "$e_1 \text{ ή αλλιώς } e_2$". 
	 Χρησιμοποιήσαμε το σύμβολο της καθέτου (`$/$') αντί της μπάρας (`$|$'), που χρησιμοποιείται στις γραμματικές χωρίς συμφραζόμενα, 
	 για να τονίσουμε την ουσιώδη διαφορά ότι η επιλογή στις PEGs δεν είναι συμμετρική, αλλά βασίζεται σε προτεραιότητα.

   \item[ Άπληστη Επανάληψη:]
	 Αν το `$e$' είναι μία έκφραση, τότε το `$(e^*)$' είναι μία έκφραση της οποίας η ερμηνεία είναι η εξής:
	 "Εφάρμοσε την έκφραση $e$ επανειλλημένα στην είσοδο, καταναλώνοτας την είσοδο προοδευτικά με κάθε επανάληψη όσο συνεχίζει να επιτυγχάνει.
	 Με την πρώτη αποτυχία, κατανάλωσε όλη την είσοδο που είχε αναγνωριστεί μέχρι τότε και επίστρεψε επιτυχώς.
	 Αν το $e$ δεν πέτυχε ούτε μία φορά, τότε επίστρεψε όπως και να 'χει επιτυχώς χωρίς να καταναλώσεις τίποτα."

   \item[ Άπληστη Θετική Επανάληψη:] %  TODO: check trans 
	 Αν το `$e$' είναι μία έκφραση, τότε το `$(e^+)$' είναι μία έκφραση της οποίας η ερμηνεία είναι η εξής:
	 "Εφάρμοσε την έκφραση $e$ επανειλλημένα στην είσοδο, 
	 και επίστρεψε επιτυχώς κατανλώνοντας όλη την είσοδο που είχε αναγνωριστεί όσο τουλάχιστον ένα στιγμιότυπο της $e$ πεχύχαινε
	 Με την πρώτη αποτυχία, κατανάλωσε όλη την είσοδο που έχει αναγνωριστεί μέχρι τότε και επίστρεψε επιτυχώς.
	 Αν το $e$ δεν πέτυχε ούτε μία φορά, τότε απότυχε χωρίς να καταναλώσεις τίποτα."

   \item[ Προαιρετικό:] 
	 Αν το `$e$' είναι μία έκφραση, τότε το `$(e?)$' είναι μία έκφραση της οποίας η ερμηνεία είναι η εξής:
	 "Προσπάθησε να εφαρμοσεις την έκφραση $e$ στην είσοδο.
	 Αν πετύχεις, τότε κατανάλωσε το αναγνωρισμένο κείμενο και επίστρεψε επιτυχώς.
	 Αν το $e$ αποτύχει, τότε επίστρεψε επιτυχώς όπως και να `χει αλλά μην καταναλώσεις τίποτα από την είσοδο."

   \item[ Ακουλουθείται-Από Κατηγόρημα:]
	 Αν το `$e$' είναι μία έκφραση, τότε το `$\&(e)$' είναι μία έκφραση της οποίας η ερμηνεία είναι η εξής:
	 "Προσπάθησε να εφαρμοσεις την έκφραση $e$ στην είσοδο.
	 Αν πετύχεις, τότε επίστρεψε επιτυχώς αλλά μην καταναλώσεις τίποτα από την είσοδο 
	 (δηλαδή επίστρεψε στην θέση της εισόδου που ήσουν πριν εφαρμοστεί το $e$).
	 Αν το $e$ αποτύχει, τότε απότυχε".

   \item[ Δεν-Ακολουθείται-Από Κατηγόρημα:]
	 Αν το `$e$' είναι μία έκφραση, τότε το `$!(e)$' είναι μία έκφραση της οποίας η ερμηνεία είναι η εξής:
	 "Προσπάθησε να εφαρμοσεις την έκφραση $e$ στην είσοδο.
	 Αν πετύχεις, τότε απότυχε αλλά μην καταναλώσεις τίποτα από την είσοδο.
	 Αν το $e$ επιτύχει, τότε απότυχε αλλά μην καταναλώσεις τίποτα από την είσοδο."

 \end{description}

Παρά την ποικιλία των παρεχόμενων εκφράσεων, όλες αυτές μπορούν να συρρικνωθούν σε έναν μικρό "πυρήνα" από στοιχειώδεις εκφράσεις.

\section{Ένα παράδειγμα μιας PEG}

\begin{figure}[h]
    \centering
    \includegraphics[width=0.3\textwidth]{pics/peg_example}
    \caption{Μία PEG για μία απλή αριθμητική γλώσσα}
    \label{fig:peg_example}
\end{figure}

Το Σχήμα \ref{fig:peg_example} παρουσιάζει μία PEG γραμματική για μία απλοϊκή αριθμητική γλώσσα.

\begin{figure}[h]
    \centering
    \includegraphics[width=0.5\textwidth]{pics/calc_parse}
	\caption{Αναγνωρίζοντας το `$(12-3)$' με βάση τη γραμματική στο Σχήμα \ref{fig:peg_example}}
    \label{fig:calc_parse}
\end{figure}

Το Σχήμα \ref{fig:calc_parse} απεικονίζει πώς η συμβολοσειρά `$(12-3)$' μπορεί να αναγνωριστεί σύμφωνα με τη γραμματική στο Σχήμα \ref{fig:peg_example}.

Πρακτικά, η περιγραφή της γραμματικής μας ορίζει νοηματικά τί θα έκανε ένας καθοδικός συντακτικός αναλυτής για να αναγνωρίσει μία συμβολοσειρά στην είσοδο.

Αρχικά, περιγράφουμε πώς θα αναγνώριζε της συμβολοσειρά $2 + 5 * 3$.

Με βάση τον ορισμό της γραμματικής, για να αναγνωρίσει το μη τερματικό $Ε$, ο συντακτικός αναλυτής θα προσπαθούσε πρώτα να αναγνωρίσει την έκφραση $T + E$ (μόνο αν αποτύχει η υποέκφραση αυτή θα δοκιμάσει την $Τ$).
Ακολούθως, θα πήγαινε στο $Τ$, μετά στ 

Η σειρά με την οποία μπαίνουν οι υποεκφράσεις στις διατεταγμένες επιλογές δεν είναι τυχαία.

\chapter{ Packrat Parsing }

\chapter{ Παράλληλο Packrat Parsing με Υπομνηματισμό }

\chapter{ Seith Fowler}

\chapter{ Παράλληλο Ordered Choice }

\chapter{ Παράλληλo Elastic Packrat Parsing }

\chapter{  Πειραματικά Αποτελέσματα }

\chapter{ Συμπεράσματα - Μελλοντική Δουλειά }

\englishtext

\chapter{Introduction}

\section{The Goal of this Project}

\section{The C++ Programming Language}


\begin{figure}[t]
\setlength\partopsep{-\topsep}% adjusts vertical space after the listing
\begin{minted}[frame=lines,linenos,numbersep=5pt]{c++}
#include <iostream>
int main() {
  std::cout << "Hello world!" << std::endl;
}
\end{minted}
\caption{Filled with joy, the Agha decides to write some code.%
  \label{fig:hello-english}}
\end{figure}


\section{The Structure of this Thesis}

\chapter{Theoretical Background}

\section{Programming Language Semantics}

\section{Domain Theory}

\selectlanguage{greek}


%%%  Bibliography

% You shouldn't want to include all the contents of thesis.bib
% in your bibliography (do you?)
\nocite{*}

\bibliographystyle{softlab-thesis}
\bibliography{thesis}


%%%  Appendices

\backmatter

\appendix

\chapter{Ευρετήριο συμβολισμών}

$A \rightarrow B$ : συνάρτηση από το πεδίο $A$ στο πεδίο $B$.

\chapter{Ευρετήριο γλωσσών}

\begin{description}
\item[C++:] πώς θα βγάλω λεφτά...
\item[Haskell:] η γλώσσα της ζωής μου αλλά πάνε οι μπύρες...
\item[Javascript:] χα, χα, χα...
\item[Python:] πώς θα τελειώνω για να πάω για μπύρες...
\end{description}


\chapter{Ευρετήριο αριθμών}

\begin{description}
\item[17:] ask Zachos.
\item[42:] life, the universe and everything --- ask Douglas.
\end{description}


%%%  End of document

\end{document}
