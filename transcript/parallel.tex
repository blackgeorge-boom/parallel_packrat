\chapter{ Parallel Packrat Parsing }
\label{ch:parallel}

Η βασική συνεισφορά που θέλουμε να κάνουμε στα πλαίσια της διπλωματικής, είναι να τροποποιήσουμε τη σειριακή έκδοση του packrat parser, ώστε να μπορεί να τρέξει αποδοτικότερα σε ένα πολυπύρηνο σύστημα.
Επιπλέον, θα δοκιμάσουμε τα παραλληλοποιήσουμε όχι μόνο την κλασική έκδοση του packrat, αλλά και παραλλαγών της.

To πρόβλημα εύρεσης παραλληλισμού δεν είναι καθόλου τετριμμένο εν γένει. 
Ειδικά στην περίπτωση μας, θα πρέπει να εξετάσουμε διάφορα κομμάτια του αλγορίθμου που μπορούν να μοιραστούν μεταξύ των νημάτων, καθώς και πώς τα νήματα αυτά θα προσπελάσουν τη δομή δεδομένων με τα ενδιάμεσα αποτελέσματα.
Εδώ, πρέπει να προσέξουμε ιδιαιτέρως δύο σημεία:

\begin{description}[font=$\bullet$\scshape\bfseries]
	\item Η πρόσβαση σε κάθε κελί της δομής δεδομένων πρέπει να γίνεται με \textit{αμοιβαίο αποκλεισμό (mutual exlusion)} μεταξύ των νημάτων.
	\item Θέλουμε η \textit{αναμονή (waiting)} των νημάτων να είναι όσο το δυνατόν μικρότερη.
\end{description}

Θα ξεκινήσουμε πρώτα με τον αλγόριθμο δυναμικού προγραμματισμού για το packrat parsing (DP packrat).
Ακολούθως, θα επικεντρωθούμε στον αλγόριθμο με υπομνηματισμό.
Θα δούμε τί έχει δοκιμαστεί ως τώρα από την επιστημονική κοινότητα, ενώ θα επιχειρήσουμε και δικές μας προσεγγίσεις.

\section{Παράλληλος DP packrat}

