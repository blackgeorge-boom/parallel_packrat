\chapter{Πειραματικά Αποτελέσματα}
\label{ch:results}

Θα αξιολογήσουμε τους αλγορίθμους μας με 3 αρχεία από τον πηγαίο κώδικα της Java \footnote{\url{http://hg.openjdk.java.net/jdk8/jdk8/jdk/file/tip/src/share/classes/java}}.
Το μέγεθός τους είναι μέτριο προς μεγάλο, ώστε να μπορέσουν να φανούν οι διαφορές στους χρόνους εκτέλεσης των αλγορίθμων.
Τα μεγέθη τους είναι:

\begin{itemize}
  \item \textit{Arrays.java} - 116K 
  \item \textit{BigDecimal.java} - 140K 
  \item \textit{Throwable.java} - 28K 
\end{itemize}

Οι μετρήσεις έγιναν με το \textit{std::chrono::high\_resolution\_clock} της STL.

\section{Packrat με ελαστικό κυλιόμενο παράθυρο}

Αρχικά, εκτελούμε διαδοχικά τον αλγόριθμο packrat με ελαστικό κυλιόμενο παράθυρο για διάφορα μήκη παραθύρου (w) και κατώφλια απενεργοποίησης μη τερματικών (thres).


\begin{table}[!ht]
\centering
\begin{tabular}{ c c c c c c c c} 
\toprule
\diagbox{thres}{w} & \makecell{256}& \makecell{512} & \makecell{1024} \\ 
\midrule
0  & 396 & 401 & \cellcolor{yellow!45}415 \\
16 & \cellcolor{green!45}378 & \cellcolor{green!45}378 & \cellcolor{green!45}381 \\
32 & \cellcolor{green!45}380 & \cellcolor{green!45}381 & \cellcolor{green!45}379 \\
48 & \cellcolor{green!45}379 & \cellcolor{green!45}379 & \cellcolor{green!45}374 \\
\bottomrule
\end{tabular}
  \caption{Arrays.java}
\end{table}

\begin{table}[!ht]
\centering
\begin{tabular}{ c c c c c c c c} 
\toprule
\diagbox{thres}{w} & \makecell{256}& \makecell{512} & \makecell{1024} \\ 
\midrule
0  & 341 & \cellcolor{yellow!45}356 & \cellcolor{yellow!45}358 \\
  16 & 334 & 337 & 336 \\
32 & \cellcolor{green!45}329 & \cellcolor{green!45}329 & \cellcolor{green!45}330 \\
48 & 333 & 334 & 332 \\
\bottomrule
\end{tabular}
  \caption{BigDecimal.java}
\end{table}

\begin{table}[!ht]
\centering
\begin{tabular}{ c c c c c c c c} 
\toprule
\diagbox{thres}{w} & \makecell{256}& \makecell{512} & \makecell{1024} \\ 
\midrule
0  & 53 & 51 & 55 \\
16 & 50 & 48 & 51 \\
32 & 51 & 47 & 51 \\
48 & 52 & 47 & 51 \\
\bottomrule
\end{tabular}
  \caption{Throwable.java}
\end{table}

\section{Παράλληλο packrat parsing}

\begin{table}[!ht]
\centering
\begin{tabular}{ c c c c c c c c} 
\toprule
  \diagbox{max\\depth}{expr\\limit} & \makecell{2}& \makecell{4} & \makecell{6} & \makecell{8} \\ 
\midrule
1 & \cellcolor{green!45}434 & \cellcolor{green!45}432 & \cellcolor{green!45}430 & 437 \\
  2 & \cellcolor{red!45}1278 & 451 & 446 & 446\\
\bottomrule
\end{tabular}
  \caption{Arrays.java}
\end{table}

\begin{table}[!ht]
\centering
\begin{tabular}{ c c c c c c c c} 
\toprule
  \diagbox{max\\depth}{expr\\limit} & \makecell{2}& \makecell{4} & \makecell{6} & \makecell{8} \\ 
\midrule
  1 & \cellcolor{green!45}368 & \cellcolor{green!45}369 & \cellcolor{green!45}367 & \cellcolor{green!45}369  \\
  2 & \cellcolor{red!45}2570 & 380 & 379 & 385\\
\bottomrule
\end{tabular}
  \caption{BigDecimal.java}
\end{table}

\begin{table}[!ht]
\centering
\begin{tabular}{ c c c c c c c c} 
\toprule
  \diagbox{max\\depth}{expr\\limit} & \makecell{2}& \makecell{4} & \makecell{6} & \makecell{8} \\ 
\midrule
  1 & \cellcolor{yellow!45}94 & \cellcolor{green!45}60 & \cellcolor{green!45}62 & \cellcolor{green!45}60 \\
  2 & \cellcolor{red!45}324 & 76 & 74 & 109 \\
\bottomrule
\end{tabular}
  \caption{Throwable.java}
\end{table}

\section{Τελικά Αποτελέσματα}

\begin{tabular}{ |p{4cm}||p{3cm}|p{3cm}|p{3cm}|  }
 \hline
  \multicolumn{4}{|c|}{Χρόνοι εκτέλεσης (ms)} \\
 \hline
  Αλγόριθμος Packrat& Arrays - 116K &BigDecimal - 140K &Throwable - 28K\\
 \hline
 Κλασικός & 404 & 350 & \cellcolor{green!45}46\\
  Elastic (256, 32) & \cellcolor{green!45}380 & \cellcolor{green!45}329 & 51\\
  Παράλληλος (1, 4) & 432 & 369 & 62\\
 \hline
\end{tabular}
